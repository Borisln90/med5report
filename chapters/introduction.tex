%!TEX root = ../main.tex
\section{Introduction} \label{sec:introduction}
The recent years a specific form of Artificial intelligence called Evolutionary Algorithms have seen more application, and are used in a wide range of programs. The concept has been around for many years, but it is only with modern computing capacity that we are able to apply these algorithms in more demanding programs to do more advanced tasks. An evolutionary algorithm (EA) mimics the rules and laws of natural selection for the sake of optimization, where the fittest individuals in a population forms new generations, ideally becoming stronger and stronger. Obviously this is in very general terms. EAs do not have to evolve a species or civilization. It can for instance also be used to create a perfect wing for an airplane given the right parameters, or calculate the shortest route between multiple cities, as is the case in the famous Travelling Salesman example. 

Meanwhile, in the video games industry, a subset of EAs called Genetic Algorithms (GA) are being experimentet with as well. Especially in terms of balancing systems, and creating adaptive AI opponents, such as in strategy games which have many different parameters that have to be balanced. For instance making sure that one army is not stronger than another but still have different units and abilities. Genetic algorithms are perfect for these kinds of optimizations, for finding the most optimal configuration, where testing every possible configuration manually to get the most balanced result would otherwise take weeks, months, years etc. EAs and GAs are all about feeding an algorithm the right parameters and then evaluating the individuals/solutions correctly to achieve a better and better solution for each generation the algorithm evolves. 

In this project we have chosen to implement a genetic algorithm in a classic arcade video game to see how this would alter the behaviours of the opponents in the game. Old arcade games have been used as a base for GA experimentation before, as these older games seem to lend themselves well by having relatively simple programming and non-costly performance cost on the system. The goal for this project is to see if the GA can make the game increasingly more difficult by utilizing this type of algorithm for the AI of the opponents i.e. does the GA evolve the opponents according to our expectations of the outcome of the fitness evaluation of the algorithm.



\subsection{Initial Problem Statement} \label{sec:initialproblemstatement}
Is it possible to create an artificial intelligence video game opponent, which modifies itself according to the players' performance?
