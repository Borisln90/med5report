\section{Conclusion} \label{sec:conclusion}

In the early stages of the project, we found evidence that pointed towards the possibility of creating an artificial video game opponent to modify itself according to player performance. Thus research indicated that if was indeed possible through the use of evolutionary computation which can be conducted by the use of a complex adaptive system(CAS) as specified in the investigations. Examples of applied CAS was found where genetic algorithms was used in games to improve performance in the video game Pac-man of either the ghosts or Pac-man.

We find evidence that point towards the possibility of applying genetic algorithms in Pac-man and account for the mechanics and parameters in the game. We find that the ghosts switch between modes, being chase, scatter and frightened mode.

We consider the possible aspects of the modes and find that the ghosts make use of a finite state machine to switch between the three modes.

The understanding of research lead us to the formulation of the final problem statement, where we ask if it is possible to alter some various states in a finite state machine to match player performance through the use of a genetic algorithm.

The analysis of research lead to an understanding of how a genetic algorithm worked and how to implement such algorithm. Additionally we found indications of how player performance could be defined and the specifications of the technical aspects of the original Pac-man to determine which modes and behaviours to utilize in the prototype and which aspects of the original Pac-man could be used to develop a modification of the game to use as a test bed to answer to the final problem statement.

The test conducted was based upon the specified evaluation of player performance and how we could account for method of alternating the various states of a finite state machine in order to match the players' performance.

With out documented test results, we found that we were unable to match the player performance with the specified player performance as stated in the evaluation.


We find no apparent reasoning to be able to argue that we have successfully answered to the final problem statement as we ask if it is generally possible to create some implementation of a video game AI, which alternates between the various state of a finite state machine, to match the player's performance through the use of a genetic algorithm.

Therefore we can conclude that with our current prototype, variation of game platform along with the variation of the various states of the finite state machine, definition of player performance and implementation of genetic algorithm, we were unable to answer to the final problem statement.

We do however only account for one single attempt of one single solution, whereof many variations of implementation of each of the mentioned elements will plausibly validate the possibility of successfully answer to the final problem statement.

To successfully answer to the final problem statement, alternative implementations and evaluations of each point in the final problem statement must be conducted in further research.

