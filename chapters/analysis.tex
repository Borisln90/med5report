\section{Analysis} \label{sec:analysis}


% Pacman content research


\subsection{Pacman}
\subsubsection{AI behaviour argumentation (how they do it)}
\subsubsection{Technical details/mechanics/paramters}

\textbf{About}\\
Pacman is an arcade game which was developed by the company Namco, and was first released in 1980's.

Extensive documentation and description of Pacman has been accounted for by Pittmann, 2011 \cite{Pitmann2011}.

\textbf{Purpose}
The general purpose of the game is to control Pacman through a maze while gathering dots. Once all the dots within the maze has been collected by Pacman, the player proceeds to the next level.







\textbf{Gameplay Details}\\
The purpose of the game is to navigate Pacman though a maze by simple up,down,left,right movements to pick up small dots within the maze while avoiding being eaten by four ghost that chases the Pacman throughout the length of the game.








\subsection{Player performance}
a. Identify pacman parameters to measure player performance. (from 4.B) Associate also with  previous research(other AI implementations in Pacman. What did they use to  identify “performance”?


b. Define “good” and “bad” performance based upon 5a. (assume that we use only win/lose conditions unless prior research has based performance indications upon other game parameter results)

c. Performance results. We identify “some” performance. Therefore we configure the following “alternation of ghost behavior” in “this” way.

\subsection{Genetic Algorithm(s)}

\subsubsection{GA definitions. (what it is, and how it works}
\subsubsection{GA and Pacman(how the two can be combined.(methods)}
we must identify possible implementations of "pacman parameters" to control fitness score.
