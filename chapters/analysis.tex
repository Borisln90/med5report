%!TEX root = ../main.tex

\section{Analysis} \label{sec:analysis}


% Pacman content research


\subsection{Pacman}
%!TEX root = ../../main.tex

\subsubsection{AI Pathfinding}
The ghost AI of the original Pac-Man is based on a single pathfinding algorithm which all the modes previously discussed in the investigation utilize \cite{Pittman2011}.
In order to get a better understanding of how these modes work we need to look at this pathfinding mechanism.

The maze is made up of tiles in a grid.
This grid has a size of 28 x 36 tiles, this includes everything on the screen including score counters, and life count.
The size of the individual tiles depends on the resolution of the game space.
In the case of the origin Pac-Man the screen size was 224 x 288 pixels.
That makes each tile 8 x 8 pixels in size.

The pathfinding works by finding a path from one tile to the other using only the tiles that are walkable within the maze.
Tiles outside the maze and tiles with pieces of wall within them, while still tiles in the collection of tiles, are considered illegal tiles.

\begin{figure}[!htbp]
\centering
\includegraphics[width=0.5\textwidth]{Tiles.png}
\caption{Visible space in a game of Pac-Man: \cite{Pittman2011} }
\label{fig:Pacman_visible_space}
\end{figure}

While the ghosts can only move along a path using legal tiles, their target tile can be anywhere on the grid.
In chase mode the target tile will be somewhere near the Pac-Man (depending on the ghost in question), and in scatter mode where the ghost will move towards its designated corner of the maze.
The target tile the ghost will go to when in scatter mode is fixed outside of the maze to ensure that the ghost will never get to it, resulting in the ghost circulating in the area closest to the fixed target until it has a new target.
Another fixed tile is the one in the ghost-one in the middle of the screen.
This is activated when the ghost has been eaten by the Pac-Man and is sent back to the pen.

In Figure \ref{fig:Pacman_visible_space} we can see the grid laid out over the screen of Pac-Man. The grey tiles are considered walkable and the tiles that the pathfinding works with. The eatable pellets are the white dots in the middle of those tiles.

The bulk of the pathfinding happens when the ghost reaches an intersection in the maze.
When the ghost is one tile away from an intersection, it will adjust its direction.
It works by finding the tile with the shortest distance to the target.
It looks one ahead in every direction of the intersection and chooses whichever direction has the shortest distance to the target.
When a ghost is presented with a tie, meaning an intersection where two or more directions have equal distance to the target, the ghost will default to choosing the direction in a specific order: \emph{up, left, down, right}\cite{Pittman2011}.
This means that when the ghost is in a tie situation it will always choose the an upwards direction before left, if those two were the tie directions.

\subsubsection{Technical details/mechanics/paramters}

\textbf{About}\\
Pacman is an arcade game which was developed by the company Namco, and was first released in 1980's.

Extensive documentation and description of Pacman has been accounted for by Pittmann, 2011 \cite{Pittman2011}.

\textbf{Purpose}
The general purpose of the game is to control Pacman through a maze while gathering dots. Once all the dots within the maze has been collected by Pacman, the player proceeds to the next level. There are 255 levels in total.\\

While trying to proceed to the next level by collecting dots, there are also several other incorporated features.

\begin{figure}[!h]
\centering
\includegraphics[scale=0.5]{pacman_level.png}
\caption{Pacman level screenshot: \cite{Pittman2011} }
\label{fig:Pacman}
\end{figure}


On figure \ref{fig:Pacman}, the setup for a level is displayed. The following elements in the game, within a single level, are the following:

\begin{itemize}
\item maze\\
The maze is the area of which the Pacman and ghosts navigates around throughout a level. There exists many different mazes which depends on the version of Pacman.
\item Pacman\\
Pacman is the playing agent that is navigated throughout the levels by moving up, down, left and right.
\item Ghosts\\
The ghosts, as seen on figure \ref{fig:Pacman}, is Blinky, Pinky, Inky and Clyde. They emerge from the center of the maze and chases the Pacman throughout the level, each using unique behaviours to catch him. Description of the ghost behaviours can be found in section XX.
\item dots and energizers\\
The dots within the maze functions as the foundation for winning a level in Pacman. Once all the dots are picked up/eaten by Pacman, the level is won.\\
The dots does however also grant points to the player.\\
Each dot is worth 10 points. In the original game, as stated by Pittman, 2011 \cite{Pittman2011}, there are 240 dots in total which amounts to 2400 points.

Energizers, which can be seen on figure \ref{fig:Pacman} as 4 bigger dots placed in each corner of the maze also grants points. 50 points each.
In combination, the total amount of points acquired by dots and energizers are  2600. The energizers does however also alter the state of the ghosts into a mode of which it is possible for the Pacman to eat the ghosts and thereby also acquiring points for a limited amount of time.\\
While the ghosts remain in the state of "fleeing" caused by the Pacman eating the energizer, each consecutive eaten ghost will result in twice as many points as the previus eaten ghost, whereof the first ghost gains the player 200 points.\\
If the player succesfully eats all four ghosts wile the "fleeing" state is active, the player acquire a total of 3000 points.\\

If all ghosts are consumed under the "fleeing" state of each energizer, in combination with the points from the gathered dots the combined score is 14600.
\item fruits\\
The fruits in Pacman are bonus elements that appear in the maze at certain times through a level which also adds points to the player, if collected. The fruit appears twice in each level and spawns when a certain amount of dots has been eaten.
The amount of points gained from the fruits range from anything from 100 to 5000 points, depending on the type of fruit that spawns. Also worth mentioning is that the fruit dissapear after a certain amount of time if not consumed in time.
\end{itemize}


\subsection{Player performance(PP)}

As described in the final problem statement, we try to implement a genetic algorithm that corresponds to the players' performance in Pacman.\\

To clarify, the player performance is not to be compared with the percieved difficulty of the game, and how it correlates to the players' skill.\\ Engeser, 2008 \cite{Engeser2008} discuss the aspects of flow, which is described as a balance of challenge of the game, and skill of the player. It is the state of which, the player feels challenged while also having confidence in keeping everything under control. \cite[pp. 158]{Engeser2008}\\
This terminology must not be compared with our definition of player performance.\\

The aim of identifying player performace, is to focus on the specific game parameters, mechanics and functionalities within Pacman to access the performance of the player. We observe elements within the game, which might imply some indications as to how the player is performing.\\


\subsubsection{performance indicating mechanics}



\subsection{Genetic Algorithm(s)}

\subsubsection{Description}
Genetic algorithms was invented by John Holland in 1975 and is proposed as a heuristic method, which is a method to learn by itself based upon survival of the fittest, where it is described as a  search algorithm that uses mechanisms much like evolution to improve some solutions to a given problem. \cite[pp. 20]{Sivanandam2008}

The Genetic Algorithm start with a set of plausible solutions and by the use of genetic operator alterations like reproduction, crossovers and mutations \cite{Baltzer2014}, it develops a new set of better solutions compared to the previous. By repeating this process, a population will theoretically improve until a satisfying result is found to the given problem. \cite{BCS2013}

What is noteworthy about "satisfying results" is that we search for some better solution within a set of possible solutions, also referred to as search space. It is therefore not guaranteed to find some, if any, optimal solution to the problem but rather a/some solution(s) which can be interpreted as acceptably good. \cite[pp. 20/21]{Sivanandam2008}


In genetic algorithms, populations and individuals covers the terminology of the solutions and searches thereof. A individual is to be considered a single solution to the problem and the population is the number of individuals involved. \cite[pp. 39]{Sivanandam2008}



\subsubsection{Genetic Algorithm Components}

A Genetic Algorithm consists of at least five components as described by Baltzer, 2014 \cite{Baltzer2014}. These components can be considered to be the main components in any genetic algorithm.
\begin{itemize}
\item \textbf{Chromosome / gene representation}

A representation of a population and the subordinate content, being chromosomes and genes as seen on figure \ref{fig:gene}.
The genes functions as a string of some arbitrary length of data. A chromosome is a sequence of genes, whereof the combined chromosomes serves as the population. \cite[pp. 41]{Sivanandam2008}

As mentioned above, each individual(chromosome) is a single solution to the problem at hand. The chromosomes can be considered to be \enquote{a point in the search space of candidate solutions} \cite[pp. 7]{Melanie1990} Melanie, 1990.


\begin{figure}[!h]
\centering
\includegraphics[scale=1.0]{gene_chromosomes.jpg}
\caption{gene, chromosomes and population description}
\label{fig:gene}
\end{figure}

The chromosomes which contains an array of genes is an abstract representation of data. Thus the chromosomes and genes must be encoded in some way, dependent on the problem. It is basically a method of representing the individual genes, which translates into the data.
There are several ways that the genes can be encoded. Some of the encoding methods are listed by Obitko, 1998 \cite{Marek1998} and by Sivanandam, 2008 \cite[pp. 43]{Sivanandam2008}

The types of encoding methods are as follows:
\begin{itemize}
\item Binary encoding
\item Octal encoding
\item hexadeciman encoding
\item Permutation encoding(sequence of numbers)
\item value encoding(string of values. numbers, letters, strings etc)
\item Tree encoding(every chromosome is a tree of some objects)
\end{itemize}


\item \textbf{Initialisation of the population}

A population is a collection of individuals(chromosomes).



An initial population must be created as a starting point for the algorithm. This representation of a population is chosen randomly, as there might be no prior evident solutions. Dependent on the stated problem at hand, the initialization of populations might also be carried out with some good solutions to the problem. However in most cases, the initial population is chosen at random. \cite[pp. 41/42]{Sivanandam2008}





\item \textbf{evaluation function(fitness)}

In order to improve further generations, the fitness of each chromosome in the current population must be evaluated. The genetic algorithm does so by the use of a fitness function, which assign some score(fitness) to each of the chromosomes within a specific population.
In general manners, the fitness of a given chromosome is dependant on how successively the given chromosome solved the designated problem, and is conveyed through a range of values. The better the solution, the higher the fitness score \cite[pp. 8]{Melanie1990}

how "good" the solution is, is the whole purpose of the fitness function to identify. the fitness function perform some evaluation of the acquired results and then assigns specific fitness score to that solution. How the fitness function performs the evaluation of solution is dependent on the problem itself. There are no single solutions as to how the fitness function is to be developed, but can be really be anything, that successively covers the plausible methods of evaluating the solutions. \cite[pp. 31]{Sivanandam2008}




\item \textbf{Genetic operators altering chromosomes(reproduction/selection, crossover and mutation.}


Genetic operators in some of the more simple Genetic Algorithms are reproduction/selection, crossover and mutation. Melanie, 1990. \cite{Melanie1990}

\textbf{reproduction/selection:} \enquote{This operator selects chromosomes in the population for reproduction. The fitter the chromosome, the more times it is likely to be selected to reproduce.} \cite[pp. 8]{Melanie1990}
The idea behind this operator is that preferences are given to the better chromosomes in the population, which is then passed on to the next generation. This is based upon the fitness score.

Within the selection operator, several selection methods exists which defines how the actual selection of "best" individuals is to be conducted.

Some of the selection methods are:
\begin{itemize}
\item Roulette Wheel Selection
\item Random Selection
\item Rank Selection
\item Tournament Selection
\item Boltzmann Selection
\item Stochastic Universal Sampling
\item Steady-State Selection
\item Elitism
\item Fitness Proportionate selection
\item Scaling Selection
\item Generation Selection
\item Hierachical Selection
\end{itemize}
The list is from Sivanandam, 2008 \cite[pp. 46-50]{Sivanandam2008} and Markzyk, 2004 \cite{Adam2004}.

The different selection methods will not be accounted for and described in this chapter, however the appropriate selection method will be accessed in the design chapter along with descriptive reasoning and method of application of the selected method(s), or combination of such.

\textbf{Crossover:} \enquote{This operator randomly chooses a locus and exchange the subsequences before and after that locus between two chromosomes to create two offspring.} \cite[pp. 8]{Melanie1990}

The crossover method takes two parent solutions(chromosomes) to create new individuals, by swapping genes between two of the potentially good solutions. There exists several crossover methods, which are:


\begin{itemize}
\item Single Point Crossover
\item Two Point Crossover
\item Multi-Point Crossover (N-Point crossover)
\item Uniform Crossover
\item Three Parent Crossover
\item Crossover with Reduced Surrogate
\item Shuffle Crossover
\item Precedence Preservative Crossover (PPX)
\item Ordered Crossover
\item Partially Matched Crossover (PMX)
\end{itemize}
The list is from Sivanandam, 2008 \cite[pp. 50-56]{Sivanandam2008}

Similarly with the list of selection operators, we will not account for the crossover operator methods, but will be accessed in the design chapter with similar procedures and content.

Additionally with the crossover operator, we can describe the crossover probability.
The crossover probability is a parameter that describes the likelihood of crossover. It basically states how often that the crossover method will be performed.\cite[pp. 56]{Sivanandam2008}

The crossover probability determines how the new generation(offspring) is altered. If there is some crossover probability then the offspring will contain genes from both of the parent chromosomes. If there is no crossover probability then the offspring will be copies of the parent chromosomes. They will be exact copies, but does not mean that they are the same.


\textbf{Mutation:} \enquote{This operator randomly flips some of the bits in a chromosome.} \cite[pp. 8]{Melanie1990}

The mutation operator happens after the crossover operator. The mutation randomly alters the structure(ordering of the genes) within the chromosomes or randomly modifies it.
The mutation of the chromosomes can happen in the following manners, dependent on the kind of representation of genes; flipping, interchanging and reversing.\cite[pp. 57]{Sivanandam2008}

Flipping genes/values is simply alterations of some to other, and other to some which then indicates a child from that chromosome. Example can be that a gene(1) is converted to gene(0), and gene(0) is converted to gene(1).

Interchanging is randomly assigning two genes within a chromosome and then switch locations of the two genes.

Lastly, reversing is randomly choosing a gene within a chromosome and thereafter change the location of the gene with a gene next to it.

Much like crossover probability, there is also a mutation probability. It states how often parts of the chromosome will be mutated. If a mutation occurs, then the chromosome is changed, in some way as specified above. If the mutation probability is 0 percent, then no mutation occurs and if the mutation is at 100 percent, then the whole chromosome is changed in accordance with the type of mutation that is specified. The reason to implement mutation of the chromosomes is to ensure that the genetic algorithm does not fall into some local extremes.\cite[pp. 57]{Sivanandam2008}

\item \textbf{Parameters for population size and probabilities of genetic operators.}
\end{itemize}

The population size covers the amount of solutions(chromosomes) that is in the actual population in the search space. The actual size of the population does very much depend on the problem at hand.\cite[pp. 42]{Sivanandam2008}


Another important aspect is the actual probability of the mutation and crossover operations.
It is mentioned in the appropriate sections concerning mutation and crossover what happens in in the absence(0 percent) or fully enabled(100 percent), but does not account for the most efficient percentage of the two. It is however also dependent on the problem, so finding some specific value, or range of values to a percentage of the crossover or mutation probability will not be possible to define, before the actual problem is defined along with some indications of plausible solutions.

The alterations in the probability may or may not have positive effects on the new generations of populations.

Marek, 1998\cite{Marek1998}  describes some recommendations based upon emprirical studies which plausible can be used as a reference point, but can be considered to be general and not necessarily the most optimal configurations, so alterations and adjustments must be conducted to account for the specific problem at hand.

the general recommendations are described as followed:
\begin{itemize}
\item Crossover rate: 80-95 percent.

It is however mentioned that occurrences of crossover probabilities of 60 percent was optimal in some undefined problem scenarios.
\item Mutation rate: 0.5-1.0 percent

The mutation rate are described as being optimal when very low.
\item population size

It is described that population size very much depend on encoding and size of the chromosomes. a population size of 20-30 is stated as good, but other examples of 50-100 are also accounted for as a good population size.
\end{itemize}

\subsubsection{Genetic algorithm procedure}


Now that the general usage, definition and components has been described, a short general description of the procedure as to implementation and functionality of a general algorithm can be defined.

Each iteration of the genetic algorithm process is defined as a generation. The entire set of generations is defined as a run. \cite[pp. 9]{Melanie1990}





A general genetic algorithm is here defined algorithmically by Malhotra et.al, 2011 \cite[pp. 40]{Rahul2011}, much similar is that of Sivanandam, 2008 \cite[pp. 30-31]{Sivanandam2008}
\enquote{
\begin{description}
\item[Start:] Generate population of chromosomes.
\item[Fitness:] Evaluate fitness score of each chromosome in the population.
\item[New Population:] Create new population of solutions by repeating the following steps until the new population is created.
\begin{description}
\item[a:] Select parent chromosomes based upon the fitness score.
\item[b:] Use the crossover operator on the parents to form new children.
\item[c:] Use the mutation operator to mutate the chrildren.
\item[d:] Place the children into the new population.
\end{description}
\item[Replacement:] Use the new generation in the algorithm.
\item[Test:] If the succes criteria/conditions is met, terminate the algorithm and return the solution in the population.
\item[Loop:] Return and run step 2.
\end{description}
} \cite[pp. 40]{Rahul2011}


FOR MY OWN REFERENCE(perhaps insert into glossary)
\begin{itemize}

\item Individual - Any possible solution
\item Population - Group of all individuals
\item Search Space - All possible solutions to the problem
\item Chromosome - Blueprint for an individual
\item Trait - Possible aspect of an individual
\item Allele - Possible settings for a trait
\item Locus - The position of a gene on the chromosome
\item Genome - Collection of all chromosomes for an individual

\end{itemize}



\subsubsection{GA and Pacman(how the two can be combined.(methods)}
we must identify possible implementations of "pacman parameters" to control fitness score.
