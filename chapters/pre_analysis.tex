\section{Investigation} \label{sec:preanalysus}
This chapter will provide an overview of Artificial Intelligence were it set focus on Artificial intelligence non-player characters (NPC) in games. It will describe the general behaviors and self-modifying Artificial Intelligence there exits, in order to compare it to the original arcade game Pacman (1980). 


\subsection{Initial Problem Statement} \label{sec:initialproblemstatement}
Is it possible to create an artificial intelligence opponent in Pacman, which modifies itself according to the players' performance?


\subsection{Artificial Intelligence} \label{sec:ai}
Artificial Intelligence (AI) has been, and is still one of the absolute big buzzword in the technology industry. 
The evolution of AI have become enormous over the past decade, and is not planning to stop anytime soon with all its possibilities and constantly growing field for each time it is proven possible.
  
For example as we speak is there a project in MIT trying to establish an AI with self-awareness that can interact socially with humans.  

A branch of the technology industry that have found its promises in AI is the gaming industry. Here computational opponents is created through the use of AI 
self awarness 


As the name Artificial Intelligence implies can the term AI have several meanings. It may refer to different areas of specialization such as games playing, expert systems, natural language, neural networks and robotics.  
Therefor whenever we further use the term AI in this report, are we referring to the idea of incorporating a game opponent (NPC) as opposed to a mechanical AI. 


\subsubsection{Definition of AI}

AI builds on the idea of creating an intelligent computational component with human like artifacts that react and thinks like a human, it should process as a self-operating machine that response to tasks. 

Keep in mind that self-operating is not considered the same as self-awareness which a gameplay should not possess. Gameplays is perceived as a form of entertainment and should therefor not be aware of anything beyond the scope. The purpose of the AI is to simulate a intelligent behavior that gives the opponent a believable challenge that can be overcome. 

Before we can go any further do we need to define our definition of intelligence since it does not have a solid definition that can characterize what kinds of computational procedures we want to call intelligent. 
In our case is intelligence defined as the ability to achieve fixed goals out from several observetions.
It should base the observation out from the trial and error method, trying out random moves until one is found succesful and store that information until next time the AI is placed in same problem. If that happens should the AI remember the problem from earlier and produce the successful answer immediately. 

Some would like to term this specific artificial intelligence as an intelligent Agent (IA). 


\subsubsection{General types of AI's in games}






\subsubsection{Self-Modifying AI's in games}


\subsection{Artificial Intelligence in Pacman(SOTA)}
\subsubsection{Original AI behaviour}
\subsubsection{Self-modifying AI's in Pacman / Ms. Pacman}

\subsection{Summary}
We identify that GA’s had indeed been implemented in Pacman in several manners, both where Pacman and the ghosts have altered behaviours. However, we find no evident material as to how a GA implementation which self modifies the alternations between the behaviours of the opponent (ghosts) could be successfully implemented. Therefore, we see if this modification is possible, and if it does indeed have some sort of impact on the player’s performance in any plausible way.


\subsection{Final Problem Statement} \label{sec:finalproblemstatement}
Is it possible to create an implementation of the ghosts in Pac-Man, which alternates between the fixed behaviors of the ghosts according to the player's performance through the use of a genetic algorithm?