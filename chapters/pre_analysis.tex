\section{Investigation} \label{sec:preanalysus}
This chapter will provide an overview of Artificial Intelligence were it set focus on Artificial intelligence non-player characters (NPC) in games. It will describe the general behaviors and self-modifying Artificial Intelligence there exits, in order to compare it to the original arcade game Pacman (1980). 


\subsection{Initial Problem Statement} \label{sec:initialproblemstatement}
Is it possible to create an artificial intelligence opponent in Pacman, which modifies itself according to the players' performance?


\subsection{Artificial Intelligence} \label{sec:ai}
Artificial Intelligence (AI) has been, and is still one of the absolute big buzzword in the science industry. 
The idea of giving a machine, human characteristics have been around for centuries and is still a field that demands a lot of attention, nevertheless is successful, and have therefor become a constantly growing field for each time it is proven possible.
One field with great succes is the gaming industry, who have found AI as a good entertainment source in commercial games when developing a NPC opponent. More recently have AI also found its way in educational application that trains medical and military students in possible real life high-risk situations. Here the students answer will reflect and change the rest of their situation either for the better or worse.   

As the name Artificial Intelligence implies can the term AI have several meanings. It may refer to different areas of specialization such as games playing, expert systems, natural language, neural networks and robotics.  
Therefor whenever we further use the term AI in this report, are we referring to the idea of incorporating a game opponent (NPC) as opposed to a mechanical AI. 



\subsubsection{Definition of AI}

Although numerous definitions already exist, would we still like define what we consider an AI. The main definiton will be defined thus: 

 		AI builds on the idea of creating an intelligent computational component with human like artifacts 			reacting and thinking like a human, that should process as a self-operating machine that response to 		tasks. 

When trying to define AI we would like to break the word up in its two former arguments, Artificial and Intelligence. 
We have adopted the official definition of Artificial, which state artificial is not necessarily associated with computers and technology. It just means something biological has been made by humans, imitated a natural origin and produced a copy of it. 
In gaming AI are we considering artificial an imitation of the human brain. The brain is the main component because this is were humans store personality, self-awareness and control self-operating actions, which is what scientist would like to modernize to mechanics. 
Keep in mind that we do not see self-operating as the same as self-awareness since a gameplay should not possess this action. Gameplays is perceived as a form of entertainment and should therefor not be aware of anything beyond the scope. The purpose of the AI is to simulate a intelligent behavior that gives the opponent a believable challenge that can be overcome. 

But what is intelligent behavior? Intelligence is hard to define since it does not have a solid definition that can characterize what kinds of computational procedures we want to call intelligent. 
Therefor in our case is intelligence defined as the ability to achieve fixed goals out from several observations.
It should base its observation out from the trial and error method, trying out random moves until one is found succesful and store that information until next time the AI is placed in same problem. If that happens should the AI remember the problem from earlier and produce the successful answer immediately. 

All this put together is what we call an intelligent Agent (IA). 

"An agent is anything that is capable of acting upon information it perceives. An intelligent agent is an agent capable of making decisions about how it acts based on experience. An autonomous intelligent agent is an intelligent agent that is free to choose between different actions. "

So for further reading will we refer our AI algorithm as an Agent.  


\subsubsection{General types of AI's in games}

Seek
Pursue
Flee
Evade
Arrival
Wander
Flock
Obstacle Avodiance
Path Follow






\subsubsection{Self-Modifying AI's in games}


\subsection{Artificial Intelligence in Pacman(SOTA)}

\subsubsection{Original AI behaviour}


\subsubsection{Self-modifying AI's in Pacman / Ms. Pacman}

\subsection{Summary}
We identify that GA’s had indeed been implemented in Pacman in several manners, both where Pacman and the ghosts have altered behaviours. However, we find no evident material as to how a GA implementation which self modifies the alternations between the behaviours of the opponent (ghosts) could be successfully implemented. Therefore, we see if this modification is possible, and if it does indeed have some sort of impact on the player’s performance in any plausible way.


\subsection{Final Problem Statement} \label{sec:finalproblemstatement}
Is it possible to create an implementation of the ghosts in Pac-Man, which alternates between the fixed behaviors of the ghosts according to the player's performance through the use of a genetic algorithm?