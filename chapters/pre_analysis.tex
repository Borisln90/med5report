%!TEX root = ../main.tex

\section{Investigation} \label{sec:preanalysus}
This chapter will provide an overview of Artificial Intelligence were it set focus on Artificial intelligence non-player characters (NPC) in games. It will describe existing self-modifying Artificial Intelligence, in order to compare it to the original arcade game Pacman (1980).


\subsection{Initial Problem Statement} \label{sec:initialproblemstatement}
Is it possible to create an artificial intelligence opponent in Pacman, which modifies itself according to the players' performance?

\subsection{Artificial Intelligence} \label{sec:ai}
Artificial Intelligence (AI) has been, and is still one of the absolute big buzzword in the science industry.
The idea of giving a machine, human characteristics have been around for centuries and is still a field that demands a lot of attention, nevertheless is successful, and have therefor become a constantly growing field for each time it is proven possible.
One field with great succes is the gaming industry, who have found AI as a good entertainment source in commercial games when developing a NPC opponent. More recently have AI also found its way in educational application that trains medical and military students in possible real life high-risk situations. Here the students answer will reflect and change the rest of their situation either for the better or worse.

As the name Artificial Intelligence implies can the term AI have several meanings. It may refer to different areas of specialization such as games playing, expert systems, natural language, neural networks and robotics.
Therefor whenever we further use the term AI in this report, are we referring to the idea of incorporating a game opponent (NPC) as opposed to a mechanical AI.



\subsubsection{Definition of AI}

Although numerous definitions already exist, would we still like define what we consider an AI. The main definiton will be defined thus:

		\begin{quote}
		AI builds on the idea of creating an intelligent computational component with human like artifacts 			reacting and thinking like a human, that should process as a self-operating machine that response to 		tasks.
		\end{quote}

When trying to define AI we would like to break the word up in its two former arguments, Artificial and Intelligence.
We have adopted the official definition of Artificial, which state artificial is not necessarily associated with computers and technology. It just means something biological has been made by humans, imitated a natural origin and produced a copy of it.
In gaming AI are we considering artificial an imitation of the human brain. The brain is the main component because this is were humans store personality, self-awareness and control self-operating actions, which is what scientist would like to modernize to mechanics.
Keep in mind that we do not see self-operating as the same as self-awareness since a gameplay should not possess this action. Gameplays is perceived as a form of entertainment and should therefor not be aware of anything beyond the scope. The purpose of the AI is to simulate a intelligent behavior that gives the opponent a believable challenge that can be overcome.

But what is intelligent behavior? Intelligence is hard to define since it does not have a solid definition that can characterize what kinds of computational procedures we want to call intelligent.
Therefor in our case is intelligence defined as the ability to achieve fixed goals out from several observations.
It should base its observation out from the trial and error method, trying out random moves until one is found succesful and store that information until next time the AI is placed in same problem. If that happens should the AI remember the problem from earlier and produce the successful answer immediately.

All this put together is what we call an intelligent Agent (IA).

"An agent is anything that is capable of acting upon information it perceives. An intelligent agent is an agent capable of making decisions about how it acts based on experience. An autonomous intelligent agent is an intelligent agent that is free to choose between different actions. "

So for further reading will we refer our AI algorithm as an Agent.


\subsubsection{General types of AI's in games}

Seek
Pursue
Flee
Evade
Arrival
Wander
Flock
Obstacle Avodiance
Path Follow


\subsubsection{Self-Modifying AI's in games}

While the AI's execute its code, should it be able to change decisions as described in the definition of AI in order to be intelligent. Therefor are we looking into self-modifying algorithms that can alter its own instructions during run time, based on the trial and error method.

Evolutionary Computation underlies the idea of survival of the fittest, meaning the best decisions sets the seed and run solo until a better decision are found.


A branch form evolutionary computation is the genetic algorithm


be evaluated from.


Some of the possible algorithms that can alter itselfs are:

Genetric Algorithm



good and bad
and were it can be used
fordele og ulemper
8
3 flere solutions





 \cite{Gallagher2003}

\subsection{Artificial Intelligence in Pacman(SOTA)}
%!TEX root = ../../main.tex

\subsubsection{Original AI modes}
The four enemies in Pac-Man may appear to move in a random fashion but each of the four ghosts actually move in a deterministic fashion~\autocite{Pittman2011} They all share the same behavior modes that they can switch between; \textbf{Chase}, \textbf{scatter}, and \textbf{frightened} mode.

\emph{Chase mode} is, in all its simplicity, about chasing the Pac-Man down through the maze.

In \emph{scatter mode}, the ghosts forgets the Pac-Man and heads to their corner of the maze. This mode is only be active for a few seconds before they revert to chase mode.

When the Pac-Man eats one of the four special pellets in the maze, the ghosts go into \emph{frightened mode}. In Frightened mode, the ghosts will turn dark blue, turn around and aimlessly wander the maze for a few seconds before they return to their previous mode. In frightened mode the ghosts become vulnerable (indicated by the dark blue color) and it is possible for the Pac-Man to eat the ghosts for extra points. When a ghost is eaten it will return to the box in the middle of the maze as well as resume its previous mode.

The effects of frightened mode is different depending on how far in the game the player is; as the levels progress the time spend in fright mode is shortened until they no longer become frightened (They still lose their target and wander aimlessly).

These modes are part of what is called a finite state machine~\autocite[pp.44]{Buckland2005}~\autocite{Kehoe2009}.
A finite state machine is one of the most basic concepts in video game AI programming. The machine consists of a finite set of fixed states that it can occupy(In the case of a light switch those states would be ON and OFF) at any given moment in time. It can be difined like this:

\blockquote[\autocite{Buckland2005}]{\enquote{A finite state machine is a device, or a model of a device, which has a finite number of states it can be in at any given time and can operate on input to either make transitions from one state to another or to cause an output or action to take place. A finite state machine can only be in one state at any moment in time.}}

The finite states in Pac-Man are the modes mentioned above and the transition between these modes~\autocite[pp.45]{Buckland2005}.
\subsubsection{Self-modifying AI's in Pacman / Ms. Pacman}

Contrary to the classic arcade Pacman, where the implemented AI is provided as a fixed set of behaviours within the game, regardless of player performance, a self modifying AI would take some parameters into account and apply some continuous optimization. This method of applying some sort of optimization is known as evolutionary computation.


There exists many examples of applied evolutionary algorithms in Pacman, as well as Ms. Pacman whereof the earliest examples was that of Koza in 1992.\cite{Koza1992}

This implementation of genetic programming was aimed at controlling Pac-Man himself through functions, conditions and primitives to evolve the movement of Pac-Man. \cite[pp. 2]{Lucas2005}
Research conducted by Simon M. Lucas in 2005 also tried to approach a method to evolve Ms. Pac-Man playing agents through the use of an evolutionary algorithm. \cite{Lucas2005}

With their evolutionary algorithm the objective was to evolve the best possible player, where they measured some average score over numerous games, in order to create a neural network to be evolved. \cite[pp. 8]{Lucas2005}

in the conclusions of their research, it is mentioned that whether or not the ghosts behaved deterministic or non-deterministic would have a significant effect on the evolved player. A non-deterministic version of the game would result in a much harder game  as the evolved player must learn strategies under different circumstances in the game rather than exploiting the deterministic game, where the game would have a certain way the ghosts move, which would become predictable for the evolved player.\cite[pp. 8]{Lucas2005}

Much similarly, but with the focus on ghost behaviour, is the research of Kalyanpur and Simon, 2001. They propose a method to improve the level of play of the ghosts in Pacman by the use of an evolutionary computation method that utilized a combination of genetic algorithms and neural networks. \cite{Kalyanpur2001}

amongst the conclusions of their implemented Genetic Algorithm and Neural Networks, they describe that it can indeed be applied in games such as Pacman. Additionally they describe the Genetic Algorithms are used to isolate the best strategy, represented by the fittest ghosts of a current generation of ghosts, and thereby pursue to include these characteristics of ghost behaviour onto the next generation. \cite[pp. 8]{Kalyanpur2001}

The implemented Neural Network is described to calculate the Genetic Algorithm parameters, which is used to improve the actual efficiency of their mentioned genetic reproduction within the game. Generally for the conclusions of their implementation is that there indeed is potential in the usage of evolutionary computation in games.

It is however noteworthy that their Pac-Man platform is self constructed, and does not use the original game platform as a test bed for the research in ghost behaviour.

An example of implemented Genetic Programming is also done by Brandstetter and Ahmadi, 2012. They propose an approach as to how Ms. Pacman can be controlled based on a Genetic Algorithm.
They describe how Ms. Pac Man can be controlled by retrieving information about the current game state in order to find some optimal movement direction.
With their applied method of application, the evolved Ms. Pac Man controller was able to play on a level of a good beginner human player.


++Insert ref til Brandstetter her. Aner ikke hvorfor det ikke fungerer++

\subsection{Summary}
We identify that GA’s had indeed been implemented in Pacman in several manners, both where Pacman and the ghosts have altered behaviours. However, we find no evident material as to how a GA implementation which self modifies the alternations between the behaviours of the opponent (ghosts) could be successfully implemented. Therefore, we see if this modification is possible, and if it does indeed have some sort of impact on the player’s performance in any plausible way.


\subsection{Final Problem Statement} \label{sec:finalproblemstatement}
Is it possible to create an implementation of the ghosts in Pac-Man, which alternates between the fixed behaviors of the ghosts according to the player's performance through the use of a genetic algorithm?