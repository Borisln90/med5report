
%!TEX root = ../main.tex

\section{Investigation} \label{sec:preanalysis}
We will investigate and discuss the viability of the initial problem statement from above. We will then investigate the various aspects of artificial intelligence as well as state of the art within self-modifying AIs. The chapter will also provide various delimitations as to which parts of the initial problem statement can be used for further analysis. This chapter will develop the initial problem statement into a final problem statement.


%!TEX root = ../../main.tex


\subsection{Artificial Intelligence} \label{sec:ai}
Artificial Intelligence (AI) has been, and is still one of the absolute big buzzword in the science industry.
The idea of giving a machine, human characteristics have been around for centuries and is still a field that demands a lot of attention, \cite {Buchanan2006} nevertheless is successful, \cite {Franz2014} \cite {Varkey2014} and have therefor become a constantly growing field for each time it is proven possible. (XX Kilde)

One field with great success is the gaming industry, who have found AI as a good entertainment source in commercial games when developing a NPC opponent. (XX kilde)   An NPC is a game character (or several characters) there is not controlled by a person but by the computer. \cite {Definition2014}  For example have a NPC recently been developed to train medical and military students in real life high-risk situations through a educational application. (XX Kilde)

As the name Artificial Intelligence implies can the term AI have several meanings. It may refer to different areas of specialization such as games playing, expert systems, natural language, neural networks and robotics. \cite {Vangie2014}
Therefor whenever we further use the term AI in this report, are we referring to the idea of incorporating a game opponent (NPC) as opposed to a mechanical AI.



\subsubsection{Definition of AI}

Although numerous definitions already exist, would we still like to define what we consider an AI. Our general definition will follow the Oxford dictionaries and be defined thus:

		\begin{quote}
		The theory and development of computer systems able to perform tasks normally requiring human 		intelligence, such as visual perception, speech recognition, decision-making, and translation between languages. \cite {Oxford2014}
		\end{quote}

However as stated earlier will we not investigate fields such as speech recognition, translation or visual perception. Therefore will we further break down the general definition and look at the exact meaning of AI's two former arguments, Artificial and Intelligence.

We have adopted the official definition of Artificial, which state artificial is not necessarily associated with computers and technology. \cite {artificial2014} It just means something biological has been made by humans, imitated a natural origin and produced a copy of it.

In gaming AI are we considering artificial an imitation of the human brain. The brain is the main component because this is were humans store personality, self-awareness and control self-operating actions, which is what scientist would like to modernize to mechanics. (XX Kilde)
Keep in mind that we do not see self-operating as the same as self-awareness. According to game developer Kehoe, 2009 should a gameplay should not possess this action. \cite {Kehoe2009}Gameplays is perceived as a form of entertainment and should therefor not be aware of anything beyond the scope. The purpose of the AI is to simulate a intelligent behavior that gives the opponent a believable challenge that can be overcome.

But what is intelligent behavior? Intelligence is hard to define since it does not have a solid definition that can characterize what kinds of computational procedures we want to call intelligent.  \cite {McCarthy2007} Therefore in our case is intelligence defined as the ability to achieve fixed goals out from several observations.
It should base its observation out from the trial and error method, trying out random moves until one is found succesful and store that information until next time the AI is placed in same problem.

All this put together is what we call an intelligent Agent (IA).

"An agent is anything that is capable of acting upon information it perceives. An intelligent agent is an agent capable of making decisions about how it acts based on experience. An autonomous intelligent agent is an intelligent agent that is free to choose between different actions. "  \cite {Mills2005}



\subsubsection{General types of AI's in games}

As we state in our IPS are we researching the possibility in creating an AI that modifies itself according to players' performance. We find that it is not relevant to investigate AI behavior that can not alter itself in a adaptive manner as we are looking for. For this are we not looking further into General AI in games.


\subsubsection{Self-modifying artificial intelligence in games}
\label{section:selfmodifyingai}

In order to establish knowledge as to what type of artificial intelligence can be modified, and how, the term of "self modifiable" artificial intelligence must be elaborated. We must therefore find examples and pre-existing theories of AI systems in games that is able to alter it's predefined set of behaviours through some adaptive algorithms as a sort of self modification.

As described by Booker et al, 2005 \cite{Booker2005} in a revised perspective of the work on adaptations in natural and artificial systems conducted by John Holland, it is defined that systems that utilizes behaviours such as learning or adaptation is known as complex adaptive systems(CAS).\cite[pp. 1]{Booker2005}

It is mentioned that such systems is able to change behaviours in response to the surrounding environment while also being capable of changing the rules that actually controls it's own behaviour through learning.

Complex adaptive systems, as described by Booker et al, is a growing subject of research within the field of biological inspired computing\cite{Booker2005} . Systems imitate some CAS in nature in order to solve computational problems, which is the process of conducting calculations through computer technology. They describe the following examples of biological computing aspects to be of interest within the development of CAS:

	\begin{itemize}
	\item Evolutionary Computation
	\item Neural computation
	\item Ant-Colony-inspired algorithms
	\item Immune-system
	\item Computer security
	\item Molecular Computing
	 \end{itemize}

Lansing, 2003 \cite{Lansing2003} validates that CAS are research within the fields of social sciences along with natural sciences.  The following aspects of the mentioned CAS are however only of relevance if the mentioned examples are somewhat applicable, in any way, to artificial intelligence aspects in games.\\
Of great interest and relevance lies the aspects and usability of evolutionary computation. Evolutionary computation is the method of constructing problem solving systems, which does so by the use of computational evolutionary models.\cite{Howe2010} Amongst the plausible models of evolutionary computation lies evolutionary algorithms, genetic algorithms, evolution strategies, evolutionary programming and artificial life\cite{Howe2010}\\

Lucas and Kendall, 2006 \cite{Lucas2006} describes how evolutionary computational intelligence can be applied in games as they provide  \enquote{ competitive, dynamic
environments that make ideal test beds
for computational intelligence theories,
architectures, and algorithms.}\cite[pp. 10]{Lucas2006}\\
Examples of implemented evolutionary computation in both board games and video games is concluded to, in some cases, be superior to other systems. They describe how evolutionary computation is applicable in games and mention the development of new game genres based upon evolutionary computation. Additionally, while refering to 2D arcade game of Pacman, they describe that \enquote{advantage
of working with older-style arcade games is that they
offer a sufficiently interesting challenge, while being much
easier to implement and faster to simulate.} \cite[pp 15]{Lucas2006} \\

As Lucas and Kendall has validated the usage of Pacman as a possible test bed for evolutionary computation, the relevance of examining pre existing implementations, using Pacman, is self apparent.\\
As mentioned earlier, evolutionary computation offers several models of computation. By examining the existing implementations, we can most likely distinguish one or several methods of successfully implementing evolutionary computation.

\subsubsection{Evolutionary computation in Pacman}


There exists many examples of applied evolutionary algorithms in Pacman, as well as Ms. Pacman whereof the earliest examples was that of Koza in 1992.\cite{Koza1992}

This implementation utilized genetic programming and was aimed at controlling Pac-Man himself through functions, conditions and primitives to evolve the movement of Pac-Man. \cite[pp. 2]{Lucas2005}
Research conducted by Simon M. Lucas in 2005 also tried to approach a method to evolve Ms. Pac-Man playing agents through the use of an evolutionary algorithm. \cite{Lucas2005}

With their evolutionary algorithm the objective was to evolve the best possible player, where they measured some average score over numerous games, in order to create a neural network to be evolved. \cite[pp. 8]{Lucas2005}

in the conclusions of their research, it is mentioned that whether or not the ghosts behaved deterministic or non-deterministic would have a significant effect on the evolved player. A non-deterministic version of the game would result in a much harder game  as the evolved player must learn strategies under different circumstances in the game rather than exploiting the deterministic game, where the game would have a certain way the ghosts move, which would become predictable for the evolved player.\cite[pp. 8]{Lucas2005}

Much similarly, but with the focus on ghost behaviour, is the research of Kalyanpur and Simon, 2001. They propose a method to improve the level of play of the ghosts in Pacman by the use of an evolutionary computation method that utilized a combination of genetic algorithms and neural networks. \cite{Kalyanpur2001}

amongst the conclusions of their implemented Genetic Algorithm and Neural Networks, they describe that it can indeed be applied in games such as Pacman. Additionally they describe the Genetic Algorithms are used to isolate the best strategy, represented by the fittest ghosts of a current generation of ghosts, and thereby pursue to include these characteristics of ghost behaviour onto the next generation. \cite[pp. 8]{Kalyanpur2001}

The implemented Neural Network is described to calculate the Genetic Algorithm parameters, which is used to improve the actual efficiency of their mentioned genetic reproduction within the game. Generally for the conclusions of their implementation is that there indeed is potential in the usage of evolutionary computation in games.

It is however noteworthy that their Pac-Man platform is self constructed, and does not use the original game platform as a test bed for the research in ghost behaviour.

An example of implemented Genetic Programming is also done by Brandstetter and Ahmadi, 2012. They propose an approach as to how Ms. Pacman can be controlled based on a Genetic Algorithm.
They describe how Ms. Pac Man can be controlled by retrieving information about the current game state in order to find some optimal movement direction.
With their applied method of application, the evolved Ms. Pac Man controller was able to play on a level of a good beginner human player. \cite{Brandstetter2012}

The mentioned examples are a few possible implementations of evolutionary computation in Pacman and Ms. Pacman. The combined applied methods in these examples are genetic programming, genetic algorithms, neural networks to control genetic algorithm parameters and evolutionary algorithms.\\

The difference between genetic algorithms and genetic programming is the actual representation of solutions, whereof the genetic programming is a subsection of genetic algorithms. The algorithm creates strings of data that combined represent some solution to a problem and the programming solutions are actual computer programs.\cite{genetic}
Genetic algorithms and genetic programming are subtypes of evolutionary algorithms.

What is interesting about the examples of applied evolutionary computation in Pacman is the method of application. The implementation does solely focus upon the improved solutions of the actual behaviours of either the ghost(s) or Pacman AI.\\
Even though a prominent factor of Pacman AI is the behaviour of either the ghosts or Pacman, there are certain aspects which is still not speicified in the above mentioned examples of evolutionary computation implementation. Behaviours of either the ghosts or pacman is described, but there are several other aspects of plausible application of evolutionary computation that serves as a interesting aspect of plausible application of evolutionary computation.\\

\subsection{Artificial Intelligence in Pac-man(SOTA)}
%!TEX root = ../../main.tex

\subsubsection{Original AI modes}
The four enemies in Pac-Man may appear to move in a random fashion but each of the four ghosts actually move in a deterministic fashion~\autocite{Pittman2011} They all share the same behavior modes that they can switch between; \textbf{Chase}, \textbf{scatter}, and \textbf{frightened} mode.

\emph{Chase mode} is, in all its simplicity, about chasing the Pac-Man down through the maze.

In \emph{scatter mode}, the ghosts forgets the Pac-Man and heads to their corner of the maze. This mode is only be active for a few seconds before they revert to chase mode.

When the Pac-Man eats one of the four special pellets in the maze, the ghosts go into \emph{frightened mode}. In Frightened mode, the ghosts will turn dark blue, turn around and aimlessly wander the maze for a few seconds before they return to their previous mode. In frightened mode the ghosts become vulnerable (indicated by the dark blue color) and it is possible for the Pac-Man to eat the ghosts for extra points. When a ghost is eaten it will return to the box in the middle of the maze as well as resume its previous mode.

The effects of frightened mode is different depending on how far in the game the player is; as the levels progress the time spend in fright mode is shortened until they no longer become frightened (They still lose their target and wander aimlessly).

These modes are part of what is called a finite state machine~\autocite[pp.44]{Buckland2005}~\autocite{Kehoe2009}.
A finite state machine is one of the most basic concepts in video game AI programming. The machine consists of a finite set of fixed states that it can occupy(In the case of a light switch those states would be ON and OFF) at any given moment in time. It can be difined like this:

\blockquote[\autocite{Buckland2005}]{\enquote{A finite state machine is a device, or a model of a device, which has a finite number of states it can be in at any given time and can operate on input to either make transitions from one state to another or to cause an output or action to take place. A finite state machine can only be in one state at any moment in time.}}

The finite states in Pac-Man are the modes mentioned above and the transition between these modes~\autocite[pp.45]{Buckland2005}.
%\subsubsection{Self-modifying AI's in Pacman / Ms. Pacman}

Contrary to the classic arcade Pacman, where the implemented AI is provided as a fixed set of behaviours within the game, regardless of player performance, a self modifying AI would take some parameters into account and apply some continuous optimization. This method of applying some sort of optimization is known as evolutionary computation.


There exists many examples of applied evolutionary algorithms in Pacman, as well as Ms. Pacman whereof the earliest examples was that of Koza in 1992.\cite{Koza1992}

This implementation of genetic programming was aimed at controlling Pac-Man himself through functions, conditions and primitives to evolve the movement of Pac-Man. \cite[pp. 2]{Lucas2005}
Research conducted by Simon M. Lucas in 2005 also tried to approach a method to evolve Ms. Pac-Man playing agents through the use of an evolutionary algorithm. \cite{Lucas2005}

With their evolutionary algorithm the objective was to evolve the best possible player, where they measured some average score over numerous games, in order to create a neural network to be evolved. \cite[pp. 8]{Lucas2005}

in the conclusions of their research, it is mentioned that whether or not the ghosts behaved deterministic or non-deterministic would have a significant effect on the evolved player. A non-deterministic version of the game would result in a much harder game  as the evolved player must learn strategies under different circumstances in the game rather than exploiting the deterministic game, where the game would have a certain way the ghosts move, which would become predictable for the evolved player.\cite[pp. 8]{Lucas2005}

Much similarly, but with the focus on ghost behaviour, is the research of Kalyanpur and Simon, 2001. They propose a method to improve the level of play of the ghosts in Pacman by the use of an evolutionary computation method that utilized a combination of genetic algorithms and neural networks. \cite{Kalyanpur2001}

amongst the conclusions of their implemented Genetic Algorithm and Neural Networks, they describe that it can indeed be applied in games such as Pacman. Additionally they describe the Genetic Algorithms are used to isolate the best strategy, represented by the fittest ghosts of a current generation of ghosts, and thereby pursue to include these characteristics of ghost behaviour onto the next generation. \cite[pp. 8]{Kalyanpur2001}

The implemented Neural Network is described to calculate the Genetic Algorithm parameters, which is used to improve the actual efficiency of their mentioned genetic reproduction within the game. Generally for the conclusions of their implementation is that there indeed is potential in the usage of evolutionary computation in games.

It is however noteworthy that their Pac-Man platform is self constructed, and does not use the original game platform as a test bed for the research in ghost behaviour.

An example of implemented Genetic Programming is also done by Brandstetter and Ahmadi, 2012. They propose an approach as to how Ms. Pacman can be controlled based on a Genetic Algorithm.
They describe how Ms. Pac Man can be controlled by retrieving information about the current game state in order to find some optimal movement direction.
With their applied method of application, the evolved Ms. Pac Man controller was able to play on a level of a good beginner human player.


++Insert ref til Brandstetter her. Aner ikke hvorfor det ikke fungerer++

\subsection{Summary}
To summarize the content of investigation, there are several aspects of applicable knowledge that will make it possible to further develop a specified problem statement.

\begin{itemize}
\item Artificial intelligence\\
We find a definition of artificial intelligence that is suitable for game development.

\item Evolutionary computation\\
In section \ref{sec:selfmodifyingai},we account for the self-modifying AI opponent term and find that a possible method of self-modifications can be conducted by the use of complex adaptive systems(CAS).


Within the field of biological computing, we find that evolutionary computation is indeed used in games, with numerous examples of successful implementations in the game of Pac-man. Here several examples of genetic algorithms are applied, which is a method of evolutionary computation. The genetic algorithm, are in the specified examples used to improve the performance of either the ghosts, or the actual Pac-man playing agent.

\item game delimitation\\
Additionally, as specified in section \ref{sec:selfmodifyingai} we find evidence to support that it is indeed possible to implement a self modifying AI in Pac-man, and find several examples of actual implementations of such.

Based upon the research, there are evident examples of self-modifying AI's that is able to improve the performance of the actual ghost or Pac-man AI by the use of evolutionary computation.

\item Finite State Machines\\
In section \ref{sec:aimodes} we find that Pac-Man makes use of a finite state machine to switch between the three modes of the ghost opponents. The three modes are chase mode, scatter and frightened mode. We find that the implementations of self-modifying AI's directly manipulates the ghost behaviors.

\item player performance\\
The purpose of identifying player performance is to focus on specific game parameters and mechanics within the specified game to access the performance of the player. This is done by comparing goals with the progress of attaining those said goals. As the game, along with implemented game parameters and mechanics is not yet decided and specified, the definition of player performance will be conducted later in the analysis.
\end{itemize}

With the following known aspects of investigation, it is possible to define a final problem statement as such.


\subsection{Final Problem Statement} \label{sec:finalproblemstatement}
Is it possible to create an implementation of a video game AI, which alternates between the various states of a finite state machine, according to the player's performance through the use of a genetic algorithm?
