\section{Evaluation} \label{sec:evaluation}

Introduction. Describe the content of this chapter.

\subsubsection{Purpose}
Describe the general purpose of the applied method of evaluation. We make people play and look for scores, ultimately to evaluate whether or not, not depending, on their ability, that the players should end up with a equal amount of points of our GA actually works.

\subsubsection{Null and alternative hypothesis}


As we should always to try reject null hypothsis and we hope to see no difference in the data, we switch the formulations between the two and state the following:


\textbf{Null hypothesis}


There is some significant difference between...


\textbf{Alternative hypothesis}


There is no significant difference between...


\subsubsection{Pre test}

We perform a pre test to ensure that the applied GA is actually functional. We look at the following aspects:

\begin{itemize}
\item number of generations
\item population size
\item time used to test x amount of populations/generations
\end{itemize}

We argue that we will not be able to identy whether or not our crossover and mutation has the "best" probabilities, but argue that what others have done must be sufficient. We do however get some idea, by trying the game ourself, in terms of points acquired, how many generations we must make the player play to obtain a satisfying result.


\textbf{alterations and changes to the product}
Account for what we found(results) and make the changes and argue as to why we changed what we did.

\subsubsection{Procedure}
Describe the test procedure.

\begin{itemize}
\item amount of test participants.
\item Population.
\item estimated length of test.
\end{itemize}


