%!TEX root = ../../main.tex


\subsection{Artificial Intelligence} \label{sec:ai}
Artificial Intelligence has been, and still is a big buzzword in the science industry.
The idea of giving a machine human characteristics has been around for as far back as Homer, who imagined machines bringing dinner to the gods. Since then many people have toyed with the idea of using machines to act as humans, however only recently has it been possible to play with implementations of AI.~\autocite{Buchanan2006}.

One field with great success is the video games industry, who have found simple forms of AI to be good for developing opponents for their products to the point where some form of AI is now an integrated part of almost all video games available.~\autocite{Microsoft2014}.
Artificial intelligence used in games is quite different from those used in, for instance, academics.
One difference is the relatively sparse performance budget put on video games.
Where an academic artificial intelligence can be made to run on a 1000 processor super computer for many days just to complete a task, the game AI will have to make do with the hardware available to the average consumer, while the processor is also working on everything from game-logic to graphics and sound.~\autocite{Buckland2005}.

\subsubsection{Definition of AI}
As stated in the IPS this project is solely focused on artificial intelligence in regards to video game development.
The definition of an artificial intelligence for use in video games differs from the standard definition of an artificial intelligence (see glossary) in a few key areas.

According to the standard definition, artificial intelligence is about creating a system, or multiple systems, that are able to do tasks that would ordinarily require a human intelligence. In regards to video games this means simulating the behaviour of another player~\autocite{Kehoe2009}.
The definition of artificial intelligence can be split up into two parts; \emph{artificial} and \emph{intelligence}.

We have adopted the official definition of \emph{artificial}\autocite{artificial2014}, which states that something artificial is made to imitate, or simulate, something natural.

\emph{Intelligence} is hard to define since there is no solid definition that can characterize what kinds of computational procedures we want to call intelligent~\cite{McCarthy2007}. Intelligence indicates some sort of ability of decision making.

\blockquote[{\autocite{Mills2005}}]{\enquote{An agent is anything that is capable of acting upon information it perceives. An intelligent agent is an agent capable of making decisions about how it acts based on experience. An autonomous intelligent agent is an intelligent agent that is free to choose between different actions.}}

Based on this, an artificial intelligence for uses in video games is a system, capable of making decisions, simulating the behaviors of another player in the game.
This means that the key part of an artificial intelligence for use in games is the \emph{artificial} aspect of the system. The intelligence of the system can vary \enquote{The system can be as simple as a rules-based system or as complex as a system designed to challenge a player as the commander of an opposing army.}\autocite{Kehoe2009}


\subsubsection{Self-Modifying AI's in games}\label{sec:selfmodifyingai}

In order to establish knowledge as to what type of artificial intelligence can be modified, and how, the term of "self modifiable" artificial intelligence must be elaborated. We must therefore find examples and pre-existing theories of AI systems in games that is able to alter it's predefined set of behaviors through some adaptive algorithms as a sort of self modification.

As described by Booker et al, 2005 \cite{Booker2005} in a revised perspective of the work on adaptations in natural and artificial systems conducted by John Holland, it is defined that systems that utilize behaviors such as learning or adaptation is known as complex adaptive systems(CAS).\cite[pp. 1]{Booker2005}

It is mentioned that such systems are able to change behaviors in response to the surrounding environment while also being capable of changing the rules that actually controls its own behavior through learning.

Complex adaptive systems, as described by Booker et al. is a growing subject of research within the field of biological inspired computing\cite{Booker2005} . Systems imitate some CAS in nature in order to solve computational problems, which is the process of conducting calculations through computer technology. They describe the following examples of biological computing aspects to be of interest within the development of CAS:

	\begin{itemize}
	\item Evolutionary Computation
	\item Neural computation
	\item Ant-Colony-inspired algorithms
	\item Immune-system
	\item Computer security
	\item Molecular Computing
	 \end{itemize}

Lansing, 2003 \cite{Lansing2003} validates that CAS are research within the fields of social sciences along with natural sciences.  The following aspects of the mentioned CAS are however only of relevance if the mentioned examples are somewhat applicable, in any way, to artificial intelligence aspects in games.

Of great interest and relevance lies the aspects and usability of evolutionary computation. Evolutionary computation is the method of constructing problem solving systems, which does so by the use of computational evolutionary models.\cite{Howe2010} Amongst the plausible models of evolutionary computation lies evolutionary algorithms, genetic algorithms, evolution strategies, evolutionary programming and artificial life.~\cite{Howe2010}.

Lucas and Kendall, 2006~\cite{Lucas2006} describes how evolutionary computational intelligence can be applied in games as they provide \enquote{ competitive, dynamic environments that make ideal test beds for computational intelligence theories, architectures, and algorithms.}~\cite[pp. 10]{Lucas2006}
Examples of implemented evolutionary computation in both board games and video games is concluded to, in some cases, be superior to other systems. They describe how evolutionary computation is applicable in games and mention the development of new game genres based upon evolutionary computation. Additionally, while referring to 2D arcade game of Pac-Man, they explain that an \enquote{advantage of working with older-style arcade games is that they offer a sufficiently interesting challenge, while being much easier to implement and faster to simulate.} \cite[pp 15]{Lucas2006}

As Lucas and Kendall have validated the use of Pac-Man as a possible test environment for evolutionary computation, the relevance of examining pre existing implementations, using Pac-Man, is self-apparent.

As mentioned earlier, evolutionary computation offers several models of computation. By examining the existing implementations, we can most likely distinguish one or several methods of successfully implementing evolutionary computation.

\subsubsection{Evolutionary computation in Pac-man}


There exists many examples of applied evolutionary algorithms in Pac-Man, as well as Ms. Pac-Man where the earliest example was that of Koza in 1992.\cite{Koza1992}

This implementation utilized genetic programming and was aimed at controlling Pac-Man himself through functions, conditions and primitives to evolve the movement of Pac-Man. \cite[pp. 2]{Lucas2005}
Research conducted by Simon M. Lucas in 2005 also tried to approach a method to evolve Ms. Pac-Man playing agents through the use of an evolutionary algorithm. \cite{Lucas2005}

With their evolutionary algorithm the objective was to evolve the best possible player, where they measured some average score over numerous games, in order to create a neural network (EXPLAIN) to be evolved. \cite[pp. 8]{Lucas2005}

In the conclusions of their research, it is mentioned that whether or not the ghosts behaved deterministic or non-deterministic would have a significant effect on the evolved player. A non-deterministic version of the game would result in a much harder game as the evolved player must learn strategies under different circumstances in the game rather than exploiting the deterministic behaviour, where the game would have a certain way the ghosts move, which would become predictable for the evolved player.\cite[pp. 8]{Lucas2005}

Similarly, but with the focus on ghost behavior, is the research of Kalyanpur and Simon, 2001. They propose a method to improve the level of play of the ghosts in Pac-Man by the use of an evolutionary computation method that utilized a combination of genetic algorithms and neural networks. \cite{Kalyanpur2001}

Amongst the conclusions of their implemented Genetic Algorithm and Neural Networks, they describe that it can indeed be applied in games such as Pac-man. Additionally they describe the Genetic Algorithms are used to isolate the best strategy, represented by the fittest ghosts of a current generation of ghosts, and thereby pursue to include these characteristics of ghost behavior onto the next generation. \cite[pp. 8]{Kalyanpur2001}

The implemented Neural Network is described to calculate the Genetic Algorithm parameters, which is used to improve the actual efficiency of their mentioned genetic reproduction within the game. Generally for the conclusions of their implementation is that there indeed is potential in the usage of evolutionary computation in games.

It is however noteworthy that their Pac-Man platform is self constructed, and does not use the original game platform as a test environment for the research in ghost behavior.

An example of implemented Genetic Programming is also done by Brandstetter and Ahmadi, 2012. They propose an approach to how Ms. Pac-Man can be controlled based on a Genetic Algorithm.
They describe how Ms. Pac Man can be controlled by retrieving information about the current game state in order to find some optimal movement direction.
With their applied method of application, the evolved Ms. Pac-Man controller was able to play on a level of a good beginner human player. \cite{Brandstetter2012}

The mentioned examples are a few possible implementations of evolutionary computation in Pac-Man and Ms. Pac-Man. The combined applied methods in these examples are genetic programming, genetic algorithms, neural networks to control genetic algorithm parameters and evolutionary algorithms.

The difference between genetic algorithms and genetic programming is the actual representation of solutions, where the genetic programming is a subsection of genetic algorithms. The algorithm creates strings of data that combined represent some solution to a problem and the programming solutions are actual computer programs.\cite{genetic}
Genetic algorithms and genetic programming are subtypes of evolutionary algorithms.

What is considered interesting about the examples of applied evolutionary computation in Pac-Man is the method of application. The implementation does solely focus upon the improved solutions of the actual behaviors of either the ghost(s) or Pac-Man AI.

Even though a prominent factor of Pac-man AI is the behavior of either the ghosts or Pac-Man, there are certain aspects which are still not specified in the above mentioned examples of evolutionary computation implementation. Behaviors of either the ghosts or Pac-Man is described, but there are several other aspects of plausible application of evolutionary computation that serves as an interesting aspect of plausible application of evolutionary computation.