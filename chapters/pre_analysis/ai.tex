%!TEX root = ../../main.tex


\subsection{Artificial Intelligence} \label{sec:ai}
Artificial Intelligence has been, and still is a big buzzwords in the science industry.
The idea of giving a machine human characteristics has been around for as far back as Homer, who imaginged machines bringing dinner to the gods, Since then many people have toyed with the idea of using machines to act as humans, however only recently has it been possible to play with implementations of AI~\autocite{Buchanan2006}.

One field with great success is the video games industry, who have found simple forms of AI to be good for developing opponents for their products to the point where some form of AI is now an integrated part of almost all video games available. (XX kilde).
Artificial inteligences that are used in games are quite different than those used in, for instance, academics.
One difference is the relatively sparse performance budget put on video games.
Where an academic artificial inteligence can be made to run on a 1000 processor super computer for many days just to complete a task, the game AI will have to make due with the hardware available to the average consumer, while the processor is also working on everything from game-logic to graphics and sound~\autocite{Buckland2005}.

\subsubsection{Definition of AI}
As stated in the IPS this project is solely focused on artificial inteligence in regards to video game development.
The definintion of an artificial inteligence for use in video games differs from the standard definition of an artificial inteligence (see glossary) in a few key areas.

According to the standard definition, artificial inteligence is about creating a system, or multiple systems, that is able to do tasks that would ordinarily require a human inteligence. In regards to video games this means simulating the bahavior of another player~\autocite{Kehoe2009}.
You can split the definition of artificial inteligence up into two parts; \emph{artificial} and \emph{inteligence}.

We have adopted the official definition of \emph{artificial}\autocite{artificial2014}, which states that something artificial is made to imitate, or simulate, something natural.

\emph{Intelligence} is hard to define since there are no solid definition that can characterize what kinds of computational procedures we want to call intelligent~\cite{McCarthy2007}. Inteligence indicates some sort of ability of decision making.

\blockquote[{\autocite{Mills2005}}]{\enquote{An agent is anything that is capable of acting upon information it perceives. An intelligent agent is an agent capable of making decisions about how it acts based on experience. An autonomous intelligent agent is an intelligent agent that is free to choose between different actions.}}

Based on this, an artificial inteligence for uses in video games is a system, capable of making decisions, simulating the behaviors of another player in the game.
This means that the key part of an artificial inteligence for use in games is the \emph{artificial} aspect of the system. The inteligence of the system can vary \enquote{The system can be as simple as a rules-based system or as complex as a system designed to challenge a player as the commander of an opposing army.}\autocite{Kehoe2009}


\subsubsection{Self-Modifying AI's in games}

While the AI's execute its code, should it be able to change decisions as described in the definition of AI in order to be intelligent. Therefor are we looking into self-modifying algorithms that can alter its own instructions during run time, based on the trial and error method.

Complex Adaptive systems is a field that have investigated the options in imitating aspects of biological systems and implement them computational as stated by Holland, 1975. \cite {Holland1975}. The computation is tasked to solve problems corresponding to the natures way of solving, whereas some of the adaptive systems include:

	\begin{itemize}
	\item Evolutionary Computation
	\item Neural computation
	\item Ant-Colony-inspired algorithms
	\item Immune-system
	\item Computer security
	\item Molecular Computing
	 \end{itemize}


Evolutionary Computation underlies the idea of survival of the fittest, out from four significant processes.
Reproduction, competition, mutation and selection. \cite{Fogel1997} The trial and error method takes basis of these four processes in able to produce succesful answers out from an evolutionary algorithm (EA).


Reproduction wants to recombine succesful genes there have survival potential, and keep on creating an population thats getting better and better for each offspring. \cite{Fogel1997}

Selection distinguish between the parent and the offsprings quality for survival. It selects either the parent or offspring for being allowed in the next generation, typically done through their ability to solve problems and their age.   \cite{Smith2007}

Mutation is applied to the parent through input, and will deliver offspring there is slightly modified mutanted. Here the offspring depends on random choices out from a probability distribution.  \cite{Smith2007}


Competition occurs when the population is expanding in a limited resource space, and both the parent and offspring has to fight for survival. Here it is the selection process who determine who lives. \cite{Fogel1997}



A possible algorithm that make use of these four processes are the Genetic Algorithm (GA). The GA make use of the evolutionary algorithm and expand it to an algorithm that controls the best fit solution the EA is supposed to solve. Naturally we want to keep the best surviving solution, while abandoning the rest and this can be done through a fitness score that determinate a rank depending on ther suitability for the problem to be solved. \cite {Sivanandam2008}

According to a list created by Deepa, 2008 can it both have advantages and disadvantages when using a GA in games.
For example does "a GA have a wider solution space" (depending on the resources), that can differ accordingly to how many branches the parent have created. This means there can be several solutions the GA can pick from and choose the best fit option it finds succesful.
A second advantage is "it can easily be modified for different problems" because it can make use of the mutation process.
Nevertheless is the GA not total problem free and have some disadvantages as well. "There might be problems in identifying the fitness score," and "defining the problem that needs to be solved."



were it been used





