\subsubsection{Self-modifying AI's in Pacman / Ms. Pacman}

Contrary to the classic arcade Pacman, where the implemented AI is provided as a fixed set of behaviours within the game, regardless of player performance, a self modifying AI would take some parameters into account and apply some continuous optimization. This method of applying some sort of optimization is known as evolutionary computation.


There exists many examples of applied evolutionary algorithms in Pacman, as well as Ms. Pacman whereof the earliest examples was that of Koza in 1992.\cite{Koza1992}

This implementation of genetic programming was aimed at controlling Pac-Man himself through functions, conditions and primitives to evolve the movement of Pac-Man. \cite[pp. 2]{Lucas2005}
Research conducted by Simon M. Lucas in 2005 also tried to approach a method to evolve Ms. Pac-Man playing agents through the use of an evolutionary algorithm. \cite{Lucas2005}

With their evolutionary algorithm the objective was to evolve the best possible player, where they measured some average score over numerous games, in order to create a neural network to be evolved. \cite[pp. 8]{Lucas2005}

in the conclusions of their research, it is mentioned that whether or not the ghosts behaved deterministic or non-deterministic would have a significant effect on the evolved player. A non-deterministic version of the game would result in a much harder game  as the evolved player must learn strategies under different circumstances in the game rather than exploiting the deterministic game, where the game would have a certain way the ghosts move, which would become predictable for the evolved player.\cite[pp. 8]{Lucas2005}

Much similarly, but with the focus on ghost behaviour, is the research of Kalyanpur and Simon, 2001. They propose a method to improve the level of play of the ghosts in Pacman by the use of an evolutionary computation method that utilized a combination of genetic algorithms and neural networks. \cite{Kalyanpur2001}

amongst the conclusions of their implemented Genetic Algorithm and Neural Networks, they describe that it can indeed be applied in games such as Pacman. Additionally they describe the Genetic Algorithms are used to isolate the best strategy, represented by the fittest ghosts of a current generation of ghosts, and thereby pursue to include these characteristics of ghost behaviour onto the next generation. \cite[pp. 8]{Kalyanpur2001}

The implemented Neural Network is described to calculate the Genetic Algorithm parameters, which is used to improve the actual efficiency of their mentioned genetic reproduction within the game. Generally for the conclusions of their implementation is that there indeed is potential in the usage of evolutionary computation in games.

It is however noteworthy that their Pac-Man platform is self constructed, and does not use the original game platform as a test bed for the research in ghost behaviour.

An example of implemented Genetic Programming is also done by Brandstetter and Ahmadi, 2012. They propose an approach as to how Ms. Pacman can be controlled based on a Genetic Algorithm.
They describe how Ms. Pac Man can be controlled by retrieving information about the current game state in order to find some optimal movement direction.
With their applied method of application, the evolved Ms. Pac Man controller was able to play on a level of a good beginner human player.


++Insert ref til Brandstetter her. Aner ikke hvorfor det ikke fungerer++